\section{Solving large-scale eigenvalue problems}

\begin{tcolorbox}[
        title=Eigenvalue problem
    ]
    Given a matrix $A \in \mathbb{C}^{n \times n}$, find $(\lambda, \mathbf{v}) \in \mathbb{C} \times \mathbb{C}^n \setminus \{ \mathbf 0 \}$ such that
    $$
        A \mathbf{v} = \lambda \mathbf{v}
    $$
    where $\lambda$ is an eigenvalue of $A$,
    and $\mathbf v$ (non-zero) is the corresponding eigenvector.
\end{tcolorbox}

The set of all the eigenvalues of a matrix $A$ is calle the \textbf{spectrum} of $A$.

The maximum modulus of all the eigenvalues is called the \textbf{spectral radius} of $A$:
$\rho(A) = \max \{ |\lambda| : \lambda \in \lambda(A) \}$.

The problem $A \mathbf v = \lambda \mathbf v$ is equivalent to $(A - \lambda I) \mathbf v = 0$.
$\det (A - \lambda I) = 0$ is a polynomial of degree $n$ in $\lambda$:
it is called the \textbf{characteristic polynomial} of $A$ and its roots are the eigenvalues of $A$.

\subsection{Similarity transformations}

We first need to identify what types of transformations preserve eigenvalues,
and for what types of matrices the eigenvalues are easily determined.

\begin{tcolorbox}[title=Definition]
    The matrix $B$ is similar to the matrix $A$ if there exists a nonsingulat matrix $T$ such taht $B = T^{-1} A T$.
\end{tcolorbox}

With the above definition, it is trivial to show that
$$
    B \mathbf y = \lambda \mathbf y
    \implies
    T^{-1} A T \mathbf y = \lambda \mathbf y
    \implies
    A(T \mathbf y) = \lambda (T \mathbf y)
$$
so that $A$ and $B$ have the same eigenvalues,
and if $\mathbf y$ is an eigenvector of $B$,
then $\mathbf v = T\mathbf y$ is an eigenvector of $A$.