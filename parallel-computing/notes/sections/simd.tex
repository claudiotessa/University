\section{SIMD processing}

SIMD (Single Instruction Multiple Data) is a parallel processing technique that increases compute capabilities by performing the exact same operation on multiple ALUs in parallel.

the idea is to amortize the cost and complexity of managing instructions across many ALUs.

The compiler understands if loop iterations are independent, and it maps that logic onto each ALU to make use of SIMD processing capabilities within a core.

However, the fact that SIMD makes it such that all ALUs must receive the exact same command at the same time creates a problem when the code contains an \texttt{if} statement. Let's say we have 8 ALUs in a core, and 3 loop iterations will evaluate their \texttt{if} to true while 5 to false. In this case, the hardware cannot physcally branch in two directions at once, as it has only one instruction decoder.

Instead of branching, SIMD processors use a techinique called \textbf{masking} to execude the code linearly.
The processor checks the condition for all 8 elements simultaneously, then creates a \textbf{mask} of true/false values (e.g., \texttt{[T T F T F F F F]}).

When the processor broadcasts the instructions to \textit{everyone}, the \textbf{active ALUs (True)} perform the calculations, while the \textbf{inactive ALUs (False)} receive the instruction but discard the output of the calculations. If there is an \texttt{else} block, the processor flips the maps and broadcasts the instructions again.

\begin{altbox}[title = Terminology]
	\textbf{Instruction stream coherence} ("coherent execution") is a property of a program the same sequence of instructoins applies to many data elements.

	This is \textbf{necessary} for efficient SIMD processing because SIMD hardware broadcasts one instruction to all ALUs.

	It is however \textbf{not necessary} for efficient parallelization across different cores, since each core has the capability to fetch/decode a different instruction from their thread instruction stream.
\end{altbox}

\subsection{Implicit SIMD}

\textbf{Implicit SIMD} is used on modern GPUs, which means that the programmer writes standard "scalar" code, and then the compiler also generates a standard binary with scalar instructions.

The \textbf{hardware} is then responsible for simultaneously executing the same instruction from multiple program instances on different data on SIMD ALUs.

