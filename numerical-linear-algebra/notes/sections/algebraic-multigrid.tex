\section{Algebraic multigrid methods}

Algebraic multigrid (AMG) is based on MG principles but uses matrix coefficients instead of geometric information.

Classical AMG is based on the observation that the algebraic smooth error varies slowly in the direction of the matrix's relatively large (negative) coefficients.
This gives us an algebraic way to track smooth errors (however, we still need to define "large").

\begin{tcolorbox}[title = Definition: strong connection]
	Given a threshold $\theta \in (0, 1)$, we say that $i$ is \textbf{strongly connected} with $j$ if
	$$-a_{i,j} \geq \theta \max_{k \neq i}(-a_{i,k})$$
	Let us denote with $S_i$ the set of vertices that $i$ is strongly connected to by
	$$S_i = \{ j \in N_i : i \text{ strongly connects to } j \}$$
	where $N_i = \{ j \neq i : a_{i, j} \neq 0 \}$. This gives us a strength matrix $S$, with $S_i$ as its $i$-th row.
\end{tcolorbox}

\subsection{Standard coarsening}

Since algebraic smooth error varies slowly in the direction of strong connections, we should essentially \text{coarsen in the direction of strong connections}.

This results in a C/F-splitting such that F-vertices strongly connect to neighbouring C-vertices, and then the idea is to represent the values of a smooth vector at F-vertices as a linear combination of the values at C-vertices.

Given a strength matrix $S$, a general coarsening procedure consists of two steps.

\begin{algorithm}
	\caption{General coarsening algorithm}
	\begin{algorithmic}[1]
		\State Choose an independent set of C-vertices based on the graph of $S$.
		\State Choose additional C-vertices in order to sarisfy the interpolation requirements.
	\end{algorithmic}
\end{algorithm}

For standard coarsening, the strength matrix $S$ is constructed based on $A$ directly, and the general coarsening algorithm is applied.

In step \texttt{1} of the algorithm, the independent set can be chosen in the following way:

\begin{enumerate}
	\item Start with a first vertex $i$ to become a C-vertex
	\item Then, all vertices $j$ which $i$ strongly connect to become F-vertices.
	\item Next, another vertex from undecided vertices becomes a C-vertex, and all vertices which are strongly connected to it become F-vertices.
	\item This procedure is repeated until all vertices become C or F-vertices.
\end{enumerate}

Remark: this procedure highly depends on the order of the vertices. To obtain a relatively uniform distribution of C/F vertices, a suitable measure of importance is introduced (C-AMG)

\subsection{Direct interpolation}

Let $\mathbf e = (e_1, e_2, \dots, e_i, \dots)$ be the error. A simple characterization of algebraic smooth error is $A\mathbf e \approx 0$. In other words,
\begin{equation}
	\label{eq:interpolation}
	a_{i, i} e + \sum_{j \in N_i}{a_{i, j} e_j \approx 0} \quad i \in F
\end{equation}
The idea is that we want to choose proper $w_{i,j}$ such that for any algebraic smooth errors
$$
	e_i \approx \sum_{j \in C} w_{i, j} e_j \quad j \in F
$$
For $i \in F$, we define:
\begin{itemize}
	\item $C_i = C \cap N_i$ : C-points strongly connected to $i$
	\item $C_i^S = C \cap S_i$
	\item $F_i^S$ : F-points strongly connected to $i$ ($F_i = F \cap N_i$
	\item $N_i^W = \dfrac {N_i} {C_i \cup F_i^S}$ : all points weakly connected to $i$
\end{itemize}
We can then rewrite \eqref{eq:interpolation} as follows (replace $\approx$ with $=$):
$$
	a_{i,i} e_i + \alpha \sum_{j \in C_i^S} a_{i, j} e_j = 0
	\qquad
	\alpha = \dfrac{\sum_{j \in N} a_{i, j}}{\sum_{j \in C_i^S} a_{i, j}}
$$
Therefore, the formula of direct interpolation is
$$
	w_{i,j} = \alpha \frac{a_{i,j}}{a_{i,i}}, \quad i \in F, j \in C_i^S,
	\qquad
	\alpha = \frac{ \sum_{j \in N_i} a_{i,j} }{ \sum_{j \in C_i^S} a_{i,j} }
$$
